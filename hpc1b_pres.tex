%%%%%%%%%%%%%%%%%%%%%%%%%%%%%%%%%%%%%%%%%
% Beamer Presentation
% LaTeX Template
% Version 1.0 (10/11/12)
%
% This template has been downloaded from:
% http://www.LaTeXTemplates.com
%
% License:
% CC BY-NC-SA 3.0 (http://creativecommons.org/licenses/by-nc-sa/3.0/)
%
%%%%%%%%%%%%%%%%%%%%%%%%%%%%%%%%%%%%%%%%%

%----------------------------------------------------------------------------------------
%	PACKAGES AND THEMES
%----------------------------------------------------------------------------------------

\documentclass{beamer}

\mode<presentation> {

% The Beamer class comes with a number of default slide themes
% which change the colors and layouts of slides. Below this is a list
% of all the themes, uncomment each in turn to see what they look like.

%\usetheme{default}
%\usetheme{AnnArbor}
%\usetheme{Antibes}
%\usetheme{Bergen}
%\usetheme{Berkeley}
%\usetheme{Berlin}
%\usetheme{Boadilla}
%\usetheme{CambridgeUS}
%\usetheme{Copenhagen}
%\usetheme{Darmstadt}
%\usetheme{Dresden}
%\usetheme{Frankfurt}
%\usetheme{Goettingen}
%\usetheme{Hannover}
%\usetheme{Ilmenau}
%\usetheme{JuanLesPins}
\usetheme{Luebeck}
%\usetheme{Madrid}
%\usetheme{Malmoe}
%\usetheme{Marburg}
%\usetheme{Montpellier}
%\usetheme{PaloAlto}
%\usetheme{Pittsburgh}
%\usetheme{Rochester}
%\usetheme{Singapore}
%\usetheme{Szeged}
%\usetheme{Warsaw}

% As well as themes, the Beamer class has a number of color themes
% for any slide theme. Uncomment each of these in turn to see how it
% changes the colors of your current slide theme.

\usecolortheme{albatross}
%\usecolortheme{beaver}
%\usecolortheme{beetle}
%\usecolortheme{crane}
%\usecolortheme{dolphin}
%\usecolortheme{dove}
%\usecolortheme{fly}
%\usecolortheme{lily}
%\usecolortheme{orchid}
%\usecolortheme{rose}
%\usecolortheme{seagull}
%\usecolo rtheme{seahorse}
%\usecolortheme{whale}
%\usecolortheme{wolverine}

%\setbeamertemplate{footline} % To remove the footer line in all slides uncomment this line
%\setbeamertemplate{footline}[page number] % To replace the footer line in all slides with a simple slide count uncomment this line

%\setbeamertemplate{navigation symbols}{} % To remove the navigation symbols from the bottom of all slides uncomment this line
}

\usepackage{graphicx} % Allows including images
\usepackage{booktabs} % Allows the use of \toprule, \midrule and \bottomrule in tables
%\usepackage{german}
\usepackage{braket}
\usepackage{bbold}
\usepackage[utf8]{inputenc}
\usepackage{wasysym}
\usepackage{hyperref}
\usepackage{tcolorbox}
\usepackage{ragged2e}
%----------------------------------------------------------------------------------------
%	TITLE PAGE
%----------------------------------------------------------------------------------------

\title[HPC 1b]{High Performance Computing 1b \\
Parallelization of a 2D Hydro Solver} % The short title appears at the bottom of every slide, the full title is only on the title page

\author[Hertig, Radonic]{Remo Hertig\\
Stephan Radonic} % Your name
\institute[UZH] % Your institution as it will appear on the bottom of every slide, may be shorthand to save space
{
\ % Your institution for the title page
}
\date{\today} % Date, can be changed to a custom date

\begin{document}

\begin{frame}
\titlepage % Print the title page as the first slide
\end{frame}

%\begin{frame}
%\frametitle{Overview} % Table of contents slide, comment this block out to remove it
%\tableofcontents % Throughout your presentation, if you choose to use \section{} and \subsection{} commands, these will automatically be printed on this slide as an overview of your presentation
%\end{frame}

%----------------------------------------------------------------------------------------
%	PRESENTATION SLIDES
%----------------------------------------------------------------------------------------

%------------------------------------------------
%\section{HYDRO Parallelization} % Sections can be created in order to organize your presentation into discrete blocks, all sections and subsections are automatically printed in the table of contents as an overview of the talk
%------------------------------------------------
%\subsection{Introduction}
%\subsection{Parallelization strategy}
%\subsection{Serial vs. Parallel Processing}
%\subsection{Scaling and Speedup}
%\subsection{High resolution example}
%
%
%
\begin{frame}
\frametitle{Introduction and physics}
\justify
Our task was to parallelize an existing C code, originally written by Prof. Romain Teyssier in Fortran, which solves the Euler equations in 2D using a Godunov scheme. A hyperbolic PDE in conservation law form is represented as
\begin{equation}
\partial_t\mathbf{U} + \nabla\cdot\mathbf{F(U)} = 0
\label{eq:bb}
\end{equation}
The euler equations are a set of equations which basically state the momentum, mass and energy conservation. Discretization on a grid yields
\begin{equation}
\mathbf{U}^{n+1}_i=\mathbf{U^n_i}+\frac{\Delta x}{\Delta t}(\mathbf{F_{i-1/2}}-\mathbf{F_{i+1/2}})
\label{eq:bc}
\end{equation}
where $\mathbf{F_{i\pm1/2}}$ are the fluxes at the cell boundaries, the Godunov scheme uses various approximations for $\mathbf{F_{i\pm1/2}}$, depending on the specific variation of the method, e.g upwind scheme, lax-friedrich, ...
\vspace{1mm}

\end{frame}
%
%
%
\begin{frame}
\frametitle{Parallelization strategy}

\end{frame}
%
%
%
\begin{frame}
\frametitle{Serial vs. Parallel Processing}
\justify
We compare the average time step durations for a single process up to approximatly 1500 parallel processses for a fixed problem size (in our case 60994 x 120). As observable in the strong scaling graph (ref figure) we get a super linear scaling up to 800 processes. 
The super linearity of the scaling can be explained with cache usage effects. \\
\vspace{2mm}
Oprimal cache memory usage only works well for rectangular shaped domains (large x small y for parallelization in x direction). We have compared how a fixed sized problem performs with diffrent $x/y$ ratios (see figure xx). We can cleary see that the performance increases with deacrasing $y/x$ size, up to a ratio, where each processes computing domain gets too small and becomes inefficient. 
\end{frame}
%
%
%
\begin{frame}
\frametitle{Scaling and Speedup}
\begin{minipage}[1\textheight]{\textwidth}
\begin{columns}[T]
\begin{column}{0.5\textwidth}
\begin{figure}
\includegraphics[width=6.75cm]{sfftrong.png}
\caption{Strong scaling for a fixed grid size of $69994\times 120$ for 1 to 1536 processes. with the time step decreasing from 4.65 to 0.009}
\end{figure}
\end{column}
\begin{column}{0.5\textwidth}
\begin{figure}
\includegraphics[width=6.75cm]{weffak.png}
\caption{Performance comparison of a fixed sized grid with varinyg $y/x$ ratio}
\end{figure}
\end{column}
\end{columns}
\end{minipage}
\end{frame}
%
%
%
\begin{frame}
\frametitle{Scaling and Speedup}
\begin{minipage}[1\textheight]{\textwidth}
\begin{columns}[T]
\begin{column}{0.5\textwidth}
\vspace{5mm}
\justify
...tzututzut
\end{column}
\begin{column}{0.5\textwidth}
\begin{figure}
\includegraphics[width=6.75cm]{speehdup.png}
\caption{}
\end{figure}
\end{column}
\end{columns}
\end{minipage}
\end{frame}
%
%
%
\begin{frame}
\frametitle{Output image}
\justify
To show the power of parallel processing we wanna show an excerpt from our high resolution image (15000 x 500) at simulation time t=600 seconds (corresponds to the 200'000th time step in our simulation)
\begin{figure}
\includegraphics[width=12cm]{thube2.png}
\end{figure}
\end{frame}
%
%
%
\end{document} 